\documentclass{beamer}
%
% Choose how your presentation looks.
%
% For more themes, color themes and font themes, see:
% http://deic.uab.es/~iblanes/beamer_gallery/index_by_theme.html
%
\mode<presentation>
{
  \usetheme{Madrid}      % or try Darmstadt, Madrid, Warsaw, ...
  \usecolortheme{beaver} % or try albatross, beaver, crane, ...
  \usefonttheme{serif}  % or try serif, structurebold, ...
  \setbeamertemplate{navigation symbols}{}
  \setbeamertemplate{caption}[numbered]
}

\usepackage{hyperref}
\usepackage[english]{babel}
\usepackage[utf8x]{inputenc}
\usepackage{pdfpages}
\usepackage{framed, color}
\definecolor{shadecolor}{rgb}{1,0.8,0.3}
\usepackage{color}

\definecolor{codegreen}{rgb}{0,0.6,0}
\definecolor{codegray}{rgb}{0.5,0.5,0.5}
\definecolor{codepurple}{rgb}{0.58,0,0.82}
\definecolor{backcolour}{rgb}{0.95,0.95,0.92}
\definecolor{pblue}{rgb}{0.13,0.13,1}
\definecolor{pgreen}{rgb}{0,0.5,0}
\definecolor{pred}{rgb}{0.9,0,0}
\definecolor{pgrey}{rgb}{0.46,0.45,0.48}

\usepackage{listings}
\lstset{language=Python,
  showspaces=false,
  showtabs=false,
  breaklines=true,
  showstringspaces=false,
  breakatwhitespace=true,
  commentstyle=\color{pgreen},
  keywordstyle=\color{pblue},
  stringstyle=\color{pred},
  basicstyle=\ttfamily,
  frame=lrbt,xleftmargin=\fboxsep,xrightmargin=-\fboxsep
}

\title[Ragazze Digitali 2019]{Ragazze Digitali A.A. 2018/2019}
\author{E. Salvucci - S. Gattucci - C. Varini}
\date{}

\AtBeginSection[]
{
  \begin{frame}<beamer>
    \frametitle{Outline}
    \tableofcontents[currentsection,currentsubsection]
  \end{frame}
}

\begin{document}

\setbeamertemplate{background}
{\includegraphics[width=\paperwidth,height=\paperheight]{ragazze_digitali.jpg}}
\begin{frame}
\end{frame}

\setbeamertemplate{background}
% Python

\begin{frame}{Cosa faremo oggi}
    \vspace{0.8cm}
      Oggi faremo qualche esercizio per esercitarci ad utilizzare i costrutti e le funzioni che abbiamo visto fino ad ora!
\end{frame}


\begin{frame}[fragile]
\frametitle{Esercizi 1/3}
	\begin{block}{Ora prova tu!}
		\textbf{1)} Crea una funzione che chiamerai come vuoi tu, che estragga un numero casuale e lo restituisca.\\
		Chiama poi questa funzione e stampa il numero estratto insieme ad un messaggio dell'utente.\\
		\textbf{2)} Crea una funzione che, dati due parametri in ingresso, restituisca il massimo tra i due. \\
		Chiama poi questa funzione passandogli due numeri che vuoi tu!\\
		\textbf{3)} Creare una funzione che, dati come parametri due numeri, stampi tutti i numeri compresi tra i due\\

	\end{block}
\end{frame}

\begin{frame}[fragile]
\frametitle{Esercizi 2/3}
\begin{block}{Ora prova tu!}
	\textbf{4)} Dichiara una variabile globale che contiene la stringa \textit{Il cane mangia dalla ciotola}.\\ Dichiarare poi una funzione restituisce il numero di vocali presenti nella stringa dichiarata globalmente.\\
	Fuori dalla funzione poi, salva il risultato in una variabile e stampa il risultato con un messaggio

	\textbf{5)} Modifica poi la funzione appena scritta eliminando la variabile globale e passando come argomento la stringa della quale calcolare il numero delle vocali.\\
	\textbf{6)} Crea una funzione simile alla funzione precedente che conterà il numero di consonanti all'interno della variabile globale.\\
	Fuori dalla funzione, salvare il risultato in una variabile e stampare una stringa che dica all'utente se la frase contiene più vocali o più consonanti.\\
\end{block}
\end{frame}


\begin{frame}[fragile]
\frametitle{Esercizi 3/3}
\begin{block}{Ora prova tu!}
	\textbf{7)}  Crea una funzione che prenda un numero in input (chiedendolo all'utente) e stampi a video se questo numero è pari o dispari.\\
	\textbf{CONSIGLIO} l'operatore \% restituisce il resto della divisione tra il numero e 2.\\
	Es: 10\%2 = 0\\
	Es: 7\%2 = 1\\
	\textbf{8)} Crea una funzione che restituisce il contrario della stringa passata come parametro.\\
	Es: cane -$>$ enac\\
	\textbf{9)} Creare una funzione che, data come parametro una stringa, restituisca la stessa nella quale sono stati eliminati tutti gli spazi\\
\end{block}
\end{frame}





\begin{frame}[fragile]
\frametitle{Esercizi sulle Liste 1}
\begin{block}{Implementa}
  \textbf{1)}Crea una variabile globale "names" e \textit{inizializzala} con una stringa formata da 3 nomi separati da uno spazio. \\
	\textbf{2)}Crea una variabile globale "myList" e \textit{inizializzala} con names presi come Lista e non più come stringa. (usa la funzione split()). \\
  \textbf{3)}Stampa la variabile myList. \\
	\textbf{4)}Aggiungi al lista di prima i primi 3 nomi che ti vengono in mente (usa la funzione append()).\\
  \textbf{5)}Stampa la variabile myList. \\

\end{block}
\end{frame}

\begin{frame}[fragile]
\frametitle{Esercizi sulle Liste 2}
\begin{block}{Implementa}
	\textbf{6)}Crea una variabile globale "myListLen" e inizializzala con la langhezza della lista (usa la funzione len()). \\
  \textbf{7)}Stampa la variabile myListLen.\\
  \textbf{8)}Stampa il primo elemento della lista.\\
  \textbf{9)}Stampa l'ultimo elemento della lista.\\
  \textbf{10)}Stampa itutti gli elementi della lista.\\
  \textbf{11)}Stampa itutti gli elementi della lista con accanto l'indice di ogni elemento.\\

\end{block}
\end{frame}


\begin{frame}

\begin{center}
	\bigskip
	Materiale rilasciato con licenza

	\textbf{\href{http://creativecommons.org/licenses/by-sa/4.0/}{Creative Commons - Attributions, Share-alike 4.0}}

	\medskip
	\includegraphics[height=0.8cm]{cc.png}
\end{center}

\end{frame}

\end{document}
