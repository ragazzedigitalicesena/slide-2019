\section{Il regno dei draghi}

\begin{frame}{Serviranno le funzioni di alcuni moduli}
    
Cosa dobbiamo fare per poter richiamare funzioni di un modulo/libreria? 
\begin{block}{random}
Utilizzeremo la funzione randint() che abbiamo già visto
\end{block}

\begin{block}{time}
Utilizzeremo la funzione sleep(forSeconds)
\end{block}

\end{frame}

\begin{frame}[fragile]
\frametitle{Il regno dei draghi - Codice}

    \begin{lstlisting}
        import random
        import time
    \end{lstlisting}

\end{frame}

\begin{frame}{Ritorniamo al gioco di oggi}
\texttt{Ti trovi in una terra piena di draghi.\newline
        Di fronte a te ci sono due grotte\newline
        In una grotta si trova un drago simpatico e socievole che condividerà con te il suo tesoro.\newline
        Nell'altra grotta c'è invece un drago affamato e vorace.\newline
        Dentro quale grotta vuoi entrare? (1 o 2)\newline}
\textbf{1\newline}
\texttt{Entri dentro la caverna 1\newline
        E' buia e spaventosa\newline
        Un enorme drago compare all'improvviso davanti a te! Apre le sue fauci e... ti inghiottisce in un batter d'occhio!\newline
        Vuoi giocare ancora?\newline}
\textbf{No}        
        
\end{frame}

\begin{frame}[fragile]
\frametitle{Il regno dei draghi - Parte 1}

\begin{block}{E' il vostro turno!}
Scrivi una funzione, chiamandola come preferisci, che stampi a video il testo introduttivo del gioco:\newline
        
        'Ti trovi in una terra piena di draghi.\newline
        Di fronte a te ci sono due grotte\newline
        In una grotta si trova un drago simpatico e socievole che condividerà con te il suo tesoro.\newline
        Nell'altra grotta c'è invece un drago affamato e vorace.\newline
        Dentro quale grotta vuoi entrare? (1 o 2)'
\end{block}

\end{frame}

\begin{frame}[fragile]
\frametitle{Il regno dei draghi - Parte 2}

\begin{block}{E' il vostro turno!}
Scrivi una funzione, chiamandola come preferisci, che chieda al giocatore di inserire il numero della grotta nella quale vuole entrare (1 o 2) e restituisca/ritorni il numero stesso della grotta scelta 

Aiuto:
\end{block}

    \begin{lstlisting}
answer = ''
while answer != '1' or answer != '2' :
    codice del ciclo while
    "ripensa al primo giorno quando chiedevamo all'utente il nome e stampavamo 'ciao nome'"
    \end{lstlisting}
\end{frame}

\begin{frame}[fragile]
\frametitle{Il regno dei draghi - Parte 3}

\begin{block}{E' il vostro turno!}
Scrivi una funzione, chiamandola come preferisci, che
    \begin{itemize}
        \item Prenda come parametro il numero della grotta scelto dal giocatore
        \item Stampi a video
            \begin{itemize}
                \item 'Ti avvicini alla grotta numero ...' (numero della grotta inserito dal giocatore 
                \item 'E' buia e spaventosa..'
                \item 'Un enorme drago compare all'improvviso davanti a te! Apre le sue fauci e...'
            \end{itemize}
        \item Dopo la stampa a video di ogni frase fai attendere 2 secondi al giocatore per aggiungere suspance utilizzando la funzione sleep del modulo time
    \end{itemize}
\end{block}

    \begin{lstlisting}
time.sleep(2) # Interrompe l'esecuzione del programma per i secondi specificati come parametro
    \end{lstlisting}
\end{frame}

\begin{frame}[fragile]
\frametitle{Il regno dei draghi - Parte 3}

\begin{block}{E' il vostro turno!}
Nella funzione della pagina precendete, dopo le stampe e i time.sleep(2)
    \begin{itemize}
        \item Aggiungi una variabile friendlyCave dandole un valore random tra 1 e 2 (utilizzando la funzione randint del modulo random come mostrato nel codice qui sotto)
        \item Se la variabile friendlyCave ha lo stesso valore del numero della grotta scelta dal giocatore allora stampa
        
        'Hai vinto il tesoro!'
        
        altrimenti stampa
        
        'Il drago ti mangia in un boccone!'
    \end{itemize}
\end{block}

    \begin{lstlisting}
friendlyCave = random.randint(1, 2)
    \end{lstlisting}
\end{frame}

\begin{frame}[fragile]
\frametitle{Il regno dei draghi - Parte 4}

\begin{block}{E' il vostro turno!}
Dopo aver inizializzato una variabile chiamata 'playAgain' con valore 'si', utilizza un ciclo while che si ripeta finche' il valore di playAgain e' la stringa 'si' e al suo interno
    \begin{itemize}
        \item Richiami la prima funzione che avete creato per stampare il testo introduttivo del gioco
        \item Crei una variabile locale con il valore della grotta scelta dal giocatore (richiamando la seconda funzione che avete creato)
        \item Controlli se il valore della grotta scelta sia lo stesso della friendlyCave (richiamando terza funzione che avete creato)
        \item Infine stampi a video la stringa 'Vuoi giocare ancora?' e riassegni/modifichi il valore della variabile playAgain con la nuova scelta del giocatore (usando la funzione input())
    \end{itemize}
\end{block}
\end{frame}

\begin{frame}{Soluzione}
    \begin{center}
        \href{https://raw.githubusercontent.com/ragazzedigitalicesena/slide-2019/master/tex/chapter_5-8/ilRegniDeiDraghi.py}{Soluzione}
    \end{center}
\end{frame}{}