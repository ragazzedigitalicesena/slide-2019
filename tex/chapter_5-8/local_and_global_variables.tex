\section{Variabili locali e variabili globali}

\begin{frame}[fragile]
\frametitle{Variabili locali e variabili globali}
    \begin{block}{Variabili locali}
        Chiamiamo \textbf{Variabile locale} qualunque variabile dichiarata all'interno di una funzione, questa esiste solo all'interno della funzione stessa
        
        Le variabili locali vengono "dimenticate" dopo che la funzione ha raggiunto il return (e quindi ha finito le sue elaborazioni)
    \end{block}
    
    \begin{lstlisting}
def welcomePerson():
    person = 'Chiara'     # Variablile locale
    print('Benvenuta ' + person)
# Anche se non c'e' il return, da qui in poi la variable non ha piu' effetto
    \end{lstlisting}
    
    \begin{lstlisting}
welcomePerson()
person = 'Sofia'
print(person) # Sofia
welcomePerson() # Benvenuta Chiara
    \end{lstlisting}
\end{frame}

\begin{frame}[fragile]
\frametitle{Variabili globali}
    \begin{block}{Variabili globali}
        Nell'esempio precedente person = 'Sofia' e' una variabile globale, ovvero una variabile che ha effetto in tutto il nostro programma
        
        Modifichiamo leggermente l'esempio:
    \end{block}
    
        \begin{lstlisting}
def welcomePerson():
    print('Benvenuta ' + person)
    \end{lstlisting}
    
    \begin{lstlisting}
person = 'Chiara'     # Variablile globale
welcomePerson() # Benvenuta Chiara
person = 'Sofia'
welcomePerson() # Benvenuta Sofia
    \end{lstlisting}
\end{frame}