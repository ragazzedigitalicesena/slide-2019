\section{Variabili}

\begin{frame}{}
	\begin{block}{Variabili}
        \begin{itemize}
            \item Le variabili sono "contenitori" di informazioni
            \item Posso \textbf{assegnare} un valore ad una variabile utilizzando il simbolo =
            \item Una variabile rappresenta un concetto - unico ma il cui valore può cambiare nel tempo
            
            Esempio:
                    
                    \lstinline{punti = 5}
            
                    accumulo punti e...
            
                    \lstinline{punti = 10}
            \item Dopo l'assegnamento non vedrò stampato nulla a video ma da quel momento la variabile "punti" avrà valore 10        
        \end{itemize}
	\end{block}

\end{frame}

\begin{frame}{Variabili}
    \begin{block}{Utilizziamo nomi ...}
        \begin{itemize}
            \item Significativi
            \item Pronunciabili
            \item Autoesplicativi
            \item Ricercabili (ad esempio è più facile cercare "cognome" rispetto a "cgnm")
            \item Che rappresentino un concetto
            \item Che appartengano al problema che si stà rappresentando
        \end{itemize}
    \end{block}
\end{frame}

%\begin{frame}{Variabili}
%    \begin{block}{"Il nostro codice deve essere...}
%        una piacevole lettura" [Cit. Prof. Mirko Viroli]
%    \end{block}
%\end{frame}

\begin{frame}
    \begin{block}{L'importanza di un buon nome per una variabile}
        \begin{itemize}
            \item Sembra facile ma non lo è.. ma fa risparmiare tempo e rende il nostro programma più comprensibile
            \item Se un nome richiede una spiegazione o un commento (che vedremo a breve) significa che non rivela il proprio intento.
            \item e = 15
            
            eta = 16
            \item
            b = 'sofia'
            
            nome = 'sofia'
        \end{itemize}
    \end{block}
\end{frame}

\begin{frame}{Variabili e Operatori}
    \begin{block}{Posso combinare tra loro variabili e operatori}
        \begin{itemize}
            \item anno = 2019
            
            anno = 2019 + 1
            \item
            anno = 2019
            
            annoDiNascita = 2005 
            
            eta = anno - annoDiNascita
            
            print(eta)
            \item e = 15
            
            eta = 16
            \item
            b = sofia
            
            nome = sofia
        \end{itemize}
    \end{block}
\end{frame}