\begin{frame}[fragile]
\frametitle{E' il vostro turno}

\begin{block}{E' il vostro turno!}
Prendiamo come riferimento il codice al \href{https://raw.githubusercontent.com/ragazzedigitalicesena/slide-2019/master/tex/chapter_19-20/collisionDetection.py}{link}, potete copiarlo

\href{https://raw.githubusercontent.com/ragazzedigitalicesena/slide-2019/master/tex/chapter_19-20/collisionDetection.py}{Download}

\vspace{5mm}
Completiamo il codice fornito secondo le indicazioni della pagine successive.
\end{block}

\end{frame}

\begin{frame}[fragile]
\frametitle{E' il vostro turno - Parte 1}
\begin{block}{E' il vostro turno!}
    \begin{itemize}
        \item Crea una variabile che contenga, inizialmente, una lista vuota (chiamandola foods)
        \item Crea anche una costante chiamata FOODSIZE che rappresentera' sia la larghezza sia l'altezza del cibo (dandole il valore che preferisci, per il giocatore avevamo usato 50)
    \end{itemize}{}
\end{block}
\end{frame}

\begin{frame}[fragile]
\frametitle{E' il vostro turno - Parte 1}
\begin{block}{E' il vostro turno!}
    \begin{itemize}
        \item Con un ciclo for (che si ripeta 20 volte, aiuto: range(20)) aggiungiamo alla lista foods  (aiuto: append(..)) dei rettangoli che rappresenteranno i "cibi" del nostro gioco.
        
        Usiamo pygame.Rect(..) come visto nell'esempio con cui abbiamo creato il rettangolo per il giocatore.
        
        La larghezza e l'altezza del rettangolo sara' FOODSIZE.
        \item Prima abbiamo usato X\_POSITION e Y\_POSITION per posizionare il rettangolo del giocatore. In questo caso usa un numero generato casualmente compreso tra 0 e la larghezza della finstra WINDOWWIDTH - FOODSIZE (con random.randint(..) come abbiamo visto)
    \end{itemize}{}
\end{block}
\end{frame}

\begin{frame}[fragile]
\frametitle{E' il vostro turno - Parte 2}
\begin{block}{E' il vostro turno!}
    Abbiamo visto come gestire eventi generati dalla pressione e dal rilascio dei tasti su, giu, destra o sinistra.
    
    Dai la possibilita' all'utente di usare \textbf{anche} le lettere \textbf{W} per muoversi verso l'alte, \textbf{S} verso il basso, \textbf{A} verso sinistra e \textbf{D} verso destra (quindi o le frecce o le lettere).
    
    \begin{lstlisting}
    event.key == K_w
    # Per la pressione del tasto W
    event.key == K_s
    # Per la pressione del tasto S
    event.key == K_a
    # Per la pressione del tasto A
    event.key == W_d
    # Per la pressione del tasto D
    \end{lstlisting}
\end{block}

\end{frame}

\begin{frame}[fragile]
\frametitle{E' il vostro turno - Parte 3}
\begin{block}{E' il vostro turno!}
    Ora vogliamo dare la possibilita' al giocatore di essere teletrasportato in una posione casuale alla pressione del tasto X
    
    Aggiungi la gestione dell'evento "pressione del tasto X" agli if di keyup.
    
    \begin{lstlisting}
# per teletrasportare il giocatore in una
# posizione random usa, dentro all'if:
player.top = random.randint(0, WINDOWHEIGHT - player.height)
player.left = random.randint(0, WINDOWWIDTH - player.width)
    \end{lstlisting}
\end{block}

\end{frame}

\begin{frame}[fragile]
\frametitle{E' il vostro turno - Parte 4}
\begin{block}{E' il vostro turno!}
    Aggiungi un "cibo" nella posizione in cui si trovava il mouse al momento della pressione del tasto sinistro (la pressione in questo caso non ci interessa, ci interessa solo il rilascio MOUSEBUTTONUP)
        
    Lo si puo' fare prendendo spunto dall'esempio in cui abbiamo aggiunto i cibi in posizione random. In questo caso li aggiungeremo nella posizione del mouse (elementi 0 e 1 della lista event.pos[0]).
\end{block}

\end{frame}

\begin{frame}[fragile]
\frametitle{E' il vostro turno - Parte 5}
\begin{block}{E' il vostro turno!}
Abbiamo visto come muovere il giocatore verso l'alto e verso il basso. Seguendo quell'esempio aggiungi il movimento verso sinistra e verso destra (le posizioni sinistra e destra del giocatore sono rispettivamente player.left e player.right) 
\end{block}

\end{frame}

\begin{frame}[fragile]
\frametitle{E' il vostro turno - Parte 6}
\begin{block}{E' il vostro turno!}
Analogamente a come abbiamo disegnato e mostrato a video il quadrato del giocatore, mostra a video/disegna i quadrati di tutti i food presenti nella lista foods
\end{block}

\end{frame}

\begin{frame}[fragile]
\frametitle{E' il vostro turno - Parte 7}
\begin{block}{E' il vostro turno!}
La funzione \textbf{ pygame.time.get\_ticks() } restituisce il tempo (in millisecondi) trascorso da quando e' stato avviato pygame (e quindi, nel nostro caso, il programma).

\vspace{5mm}
Come possiamo fare un countdown del tempo       ?
Prova a farlo stampando il tempo con una print

Ad esempio:
\begin{itemize}
    \item 10 s
    \item 9 s
    \item 8 s 
    \item ...
\end{itemize}
\end{block}

\end{frame}