\documentclass{beamer}
%
% Choose how your presentation looks.
%
% For more themes, color themes and font themes, see:
% http://deic.uab.es/~iblanes/beamer_gallery/index_by_theme.html
%
\mode<presentation>
{
  \usetheme{Madrid}      % or try Darmstadt, Madrid, Warsaw, ...
  \usecolortheme{beaver} % or try albatross, beaver, crane, ...
  \usefonttheme{serif}  % or try serif, structurebold, ...
  \setbeamertemplate{navigation symbols}{}
  \setbeamertemplate{caption}[numbered]
} 

\usepackage{hyperref}
\usepackage{tabu}
\usepackage[italian]{babel}
\usepackage[utf8]{inputenc}
\usepackage{pdfpages}
\usepackage{framed, color}
\definecolor{shadecolor}{rgb}{1,0.8,0.3}
\usepackage{color}
\usepackage[backend=bibtex, style=numeric,sorting=none]{biblatex}

\definecolor{pblue}{rgb}{0.13,0.13,1}
\definecolor{pgreen}{rgb}{0,0.5,0}
\definecolor{pred}{rgb}{0.9,0,0}
\definecolor{pgrey}{rgb}{0.46,0.45,0.48}

\usepackage{listings}
\lstset{language=Python,
  showspaces=false,
  showtabs=false,
  breaklines=true,
  showstringspaces=false,
  breakatwhitespace=true,
  commentstyle=\color{pgreen},
  keywordstyle=\color{pblue},
  stringstyle=\color{pred},
  basicstyle=\ttfamily,
  frame=lrbt,xleftmargin=\fboxsep,xrightmargin=-\fboxsep
}

\title[Ragazze Digitali 2019]{Ragazze Digitali A.A. 2018/2019}
\author{E. Salvucci - S. Gattucci - C. Varini}
\date{}

\AtBeginSection[]
{
  \begin{frame}<beamer>
    \frametitle{Outline}
    \tableofcontents[currentsection,currentsubsection]
  \end{frame}
}

\bibstyle{unsrt}
\bibliography{bibliography.bib}

\begin{document}

\setbeamertemplate{background}
{\includegraphics[width=\paperwidth,height=\paperheight]{images/ragazze_digitali.jpg}}
\begin{frame}
\end{frame}

\setbeamertemplate{background}{}

\section{Disegnamo la finestra di gioco e gli elementi al suo interno}

\begin{frame}[fragile]
    \frametitle{La finestra di gioco}
    
    \begin{lstlisting}
WINDOW_WIDTH = 400
WINDOW_HEIGHT = 400
windowSurface = pygame.display.set_mode((WINDOW_WIDTH, WINDOW_HEIGHT), 0, 32)
pygame.display.set_caption('Input')
    \end{lstlisting}
    
    \begin{block}{set\_mode(..)}
        \begin{itemize}
            \item La funzione set\_mode(resolution, flags, depth) inizializza una nuova finestra
            \item I suoi parametri:
                \begin{itemize}
                    \item resolution
                    \item flags
                    \item depth
                \end{itemize}
        \end{itemize}
    \end{block}

\end{frame}

\begin{frame}[fragile]
    \frametitle{La finestra di gioco}
    
    \begin{block}{set\_mode(..)}
        \begin{itemize}
            \item La funzione set\_mode(resolution, flags, depth) inizializza una nuova finestra
            \item I suoi parametri:
                
                \begin{itemize}
                    \item resolution: una coppia di numeri nella forma (0, 0) che rappresenta rispettivamente la larghezza e l'altezza della finestra
                    
                    \item flags: opzioni aggiuntive, alcuni esempi:
                    
                    pygame.FULLSCREEN    create a fullscreen display
                    pygame.RESIZABLE     display window should be sizeable

                    \item depth: profondita' di colori. Possiamo non preoccuparcene, se settato a 0 pygame decide il valore migliore per noi.
                \end{itemize}
        \end{itemize}
    \end{block}

\end{frame}

\begin{frame}[fragile]
    \frametitle{pygame.Rect(...)}
    disegna un rettangolo e lo memorizza nella variabile player.
    
    Quando dovremo cambiare il rettangolo che rappresenta il giocatore andremo a modificare questa variabile
\end{frame}    
    
\begin{frame}[fragile]
    \frametitle{pygame.Rect(...)}
    \begin{block}{pygame.Rect(X\_POSITION, Y\_POSITION, WIDTH, HEIGHT)}
        \begin{itemize}
            \item X\_POSITION: Posizione della coordinata x del rettangolo sulla finestra
            \item Y\_POSITION: Posizione della coordinata y del rettangolo sulla finestra
            \item WIDTH: Larghezza del rettangolo
            \item HEIGHT: Altezza del rettangolo
        \end{itemize}
    \end{block}
    \begin{lstlisting}
X_POSITION = 300
Y_POSITION = 100
PLAYER_WIDTH = 50
PLAYER_HEIGHT = 50
player = pygame.Rect(X_POSITION, Y_POSITION, 
                     PLAYER_WIDTH, PLAYER_HEIGHT)
    \end{lstlisting}
\end{frame}

\begin{frame}[fragile]
    \frametitle{pygame.Rect(...)}
    Creiamo un altro rettangolo, come abbiamo fatto per player. Questa volta piu' piccolo.
    \begin{block}{Come possiamo fare per...}
dare al nuovo rettangolo una posizione (X, Y) random?

Cosa devo scrivere al posto dei ??? (ricordiamoci di aggiungere import random all'inizio del nostro file)
    \end{block}
    \begin{lstlisting}
foodSize = 20
# In questo caso non e' una costante perche' dopo voglio modificarne il valore 
food = pygame.Rect(???, ???, foodSize, foodSize)
    \end{lstlisting}
\end{frame}

\begin{frame}[fragile]
    \frametitle{Qual e' la posizione (x, y) di food?}

    \begin{lstlisting}
print(food.x, food.y)
    \end{lstlisting}
\end{frame}

\section{Gestione degli eventi}

\begin{frame}[fragile]
Pygame puo' generare eventi in seguito a input da tastiera da parte dell'utente.

\begin{block}{Eventi}
    \begin{itemize}
        \item QUIT: Evento generato quando il giocatore chiude la finestra di gioco.
        \item KEYDOWN: Evento generato quando il giocatore tiene premuto un tasto qualsiasi.
        \item KEYUP: Evento generato quando il giocatore rilascia un tasto qualsiasi che precedentemente stava tenendo premouto.
        \item MOUSEMOTION: Evento generato quando il mouse si muove all'interno della nostra finestra.
        \item MOUSEBUTTONDOWN: Evento generato quando viene premuto un tasto del mouse all'interno della nostra finestra
        \item MOUSEBUTTONUP: Evento generato quando viene premuto un tasto del mouse all'interno della nostra finestra
    \end{itemize}
\end{block}

\end{frame}

\section{Input da tastiera}

\begin{frame}[fragile]
    \frametitle{Input da tastiera}
    Settiamo 4 variabili, che rappresentano le diverse posizioni, con valori booleani inizialmente False.
    
    Quando il giocatore usera' il tasto sinistra cambieremo il valore corrispondente in True (e faremo lo stesso per gli altri tre tasti).
    
    Quando il giocatore lascera' il tasto premuto setteremo di nuovo il valore corrispondente a False
    \begin{lstlisting}
moveLeft = False
moveRight = False                                                                                                             
moveUp = False
moveDown = False 
    \end{lstlisting}
\end{frame}

\begin{frame}[fragile]
\frametitle{Gestire l'evento KEYDOWN (dentro il while True:)}
Se l'evento scatenato e' KEYDOWN allora l'evento stesso ha un attributo chiamato "key" che indica quale tasto e' stato premuto.

\begin{lstlisting}
for event in pygame.event.get():
    if event.type == KEYDOWN:
        if event.key == K_LEFT:
            moveRight = False
            moveLeft = True
        if event.key == K_RIGHT:
            moveLeft = False
            moveRight = True
        if event.key == K_UP:
            moveDown = False
            moveUp = True
        if event.key == K_DOWN:
            moveUp = False
            moveDown = True
\end{lstlisting}
    
\end{frame}

\begin{frame}{}
\frametitle{}
			\begin{figure}
   				\includegraphics[height=5cm]{images/constant_variables_for_keyboard_keys.png}
			\end{figure}
\end{frame}

\begin{frame}[fragile]
\frametitle{Gestire l'evento KEYUP}
Se l'evento scatenato e' KEYUP il nostro quadrato si dovra' fermare. Settiamo quindi la direzione del tasto rilasciato a False.

\begin{lstlisting}
if event.type == KEYUP:
    if event.key == K_ESCAPE:
        pygame.quit()
        sys.exit()
    if event.key == K_LEFT:
        moveLeft = False
    if event.key == K_RIGHT:
        moveRight = False
    if event.key == K_UP:
        moveUp = False
    if event.key == K_DOWN:
        moveDown = False
\end{lstlisting}
\end{frame}

\section{Muovere il giocatore}

\begin{frame}[fragile]
\frametitle{Muovere il giocatore}

\begin{block}{Muovere il giocatore}
Se una delle 4 variabili, moveDown, moveUp, moveLeft o moveRight ha valore True dobbiamo muovere il nostro giocatore (memorizzato nella variabile player).

In particolare possiamo muovere il giocatore verso l'alto se questo non e' gia' arrivato all'estremo superiore della finestra, in basso se non e' arrivato a quello inferiore e cosi' via..

Per rappresentare il movimento cambieremo la posizione verso la quale ci vogliamo muovere aggiungendo o sottraendo un valore MOVE_SPEED (che rappresenta il quanto ci stiamo muovendo)
\end{block}

\begin{lstlisting}
if moveDown and player.bottom < WINDOW_HEIGHT:
    player.top = player.top + MOVE_SPEED
if moveUp and player.top > 0:
    player.top = player.top - MOVE_SPEED
\end{lstlisting}
\end{frame}

\begin{frame}[fragile]
\frametitle{Muovere il giocatore}

\begin{block}{Muovere il giocatore}
Per rappresentare il movimento cambieremo la posizione verso la quale ci vogliamo muovere aggiungendo o sottraendo un valore MOVE_SPEED (che rappresenta il quanto ci stiamo muovendo).

Dobbiamo inizializzare la costante MOVE_SPEED (all'inizio del nostro programma, dove abbiamo inizializzato le altre costanti).
\end{block}

\begin{lstlisting}
MOVE_SPEED = 6
\end{lstlisting}
\end{frame}


\begin{frame}[fragile]
    \frametitle{Muovere il giocatore}

    \begin{block}{Come possiamo fare per...}
aggiungere il movimemnto verso destra e verso sinistra?
Cosa devo scrivere al posto dei ???
    \end{block}
    \begin{lstlisting}
if moveLeft and ???:
    player.left ??? ???
if moveRight and ???:
    player.right ??? ???
    \end{lstlisting}
\end{frame}


\section{Disegnare il giocatore e il cibo}

\begin{frame}[fragile]
\frametitle{Disegnare il giocatore e il cibo}

Per adesso abbiamo solo detto, definendo player, quali sono le caratteristiche del nostro giocatore.

Non lo abbiamo ancora disegnato e mostrato a video

\begin{lstlisting}
windowSurface.fill(WHITE) # A cosa serve qui?
pygame.draw.rect(windowSurface, BLACK, player)
pygame.draw.rect(windowSurface, GREEN, food)
\end{lstlisting}

\end{frame}

\section{Collision detection}

\begin{frame}[fragile]
\frametitle{Collision detection}

\begin{block}{player.colliderec()}
Andiamo a rilevare le collisioni tra il giocatore (player) e i cibi (foods).
Utilizziamo la funzione colliderect che verifica se due rettangoli si toccano.
\end{block}
\end{frame}

\begin{frame}[fragile]
\frametitle{Collision Detection}

\begin{lstlisting}
if player.colliderect(food):
    # Faccio qualcosa.
    pygame.draw.rect(windowSurface, RED, food)
\end{lstlisting}

O se avessimo tanti "cibi" memorizzati in una lista (foods)

\begin{lstlisting}
new_foods = foods
for food in new_foods:
    if player.colliderect(food):
        foods.remove(food)
new_foods = []
\end{lstlisting}

\end{frame}

\section{Clock e display.update()}

\begin{frame}[fragile]
\frametitle{pygame.display.update()}

\begin{block}{pygame.display.update()}

Le proprieta' del nostro giocatore sono cambiate, come anche quelle dei cibi. Il giocatore e' stato spostato, alcuni cibi rimossi e alcuni aggiunti.

E' necessario aggiornare la finestra di gioco con le nuove caratteristiche di giocatore e cibi.
\end{block}

\begin{lstlisting}
    pygame.display.update()
\end{lstlisting}

\end{frame}

\begin{frame}[fragile]
\frametitle{Clock}

\begin{block}{Clock}
Inizializziamo, all'inizio del nostro programma, una variabile globale chiamata mainClock che ci aiutera' (in generale) a tenere traccia del passare del tempo.
\end{block}

\begin{lstlisting}
mainClock = pygame.time.Clock()
\end{lstlisting}
\end{frame}

\begin{frame}[fragile]
\frametitle{pygame.time.tick(..)}

\begin{block}{pygame.time.tick(..)}
come ultima riga del while utilizziamo la funzione tick(..) che, lascia scorrere del tempo tra una chiamata e l'altra della funzione stessa. Quindi, secondo l'esempio, dopo il primo passeranno 40 millisecondi prima che venga eseguito di nuovo il tick.
\end{block}

\begin{lstlisting}
mainClock.tick(40)
\end{lstlisting}

\begin{block}{}
Perche' questo?

Proviamo a togliere la riga mainClock.tick(40)
\end{block}

\end{frame}

\begin{frame}[fragile]
\frametitle{Clock}

\begin{block}{Clock}
Creiamo ora un timer che tenga traccia del tempo trascorso e lo stampi sulla console
\end{block}

\begin{lstlisting}
# All'inizio del nostro codice
startTime = pygame.time.get_ticks()

# Alla fine del nostro codice, come ultima riga del While True:
elapsedTime = pygame.time.get_ticks()
print('Tempo trascorso:',
    int(((elapsedTime - startTime)/1000)))
\end{lstlisting}
\end{frame}


\section{Input da mouse}

\begin{frame}[fragile]
\frametitle{Input da mouse}
    
    \begin{block}{Gestire eventi del mouse}
        \begin{itemize}
            \item MOUSEMOTION: Evento generato quando il mouse si muove all'interno della nostra finestra
            
            Ha degli attributi. In particolare ci sara' utile \textbf{pos}: Indica la posizione del mouse all'interno della finestra. La posizione e' rappresentata come tupla (x, y) dove x e y sono le coordinate del mouse.
            \item MOUSEBUTTONDOWN: Evento generato quando viene premuto un tasto del mouse all'interno della nostra finestra
            
            Ha degli attributi. In particolare ci sara' utile \textbf{button}: Indica quale tasto del mouse e' stato premuto.
            
            1 rappresenta il tasto sinistro, 2 il tasto centrale (se presente), 3 il tasto destro, 4 se la rotella e' stata mossa verso l'alto o 5 se mossa verso il basso.
            
            \item MOUSEBUTTONUP: Ha gli stessi attributi di MOUSEBUTTONDOWN.
        \end{itemize}
    \end{block}
    
\end{frame}

\begin{frame}[fragile]
\frametitle{Esempio di eventi provenienti dal mouse}
Dentro al for event in pygame.event.get() (che si trova dentro al while True:)
\begin{lstlisting}
if event.type == MOUSEMOTION:
    print(event.pos[0], event.pos[1])
if event.type == MOUSEBUTTONUP:
    # Aumentiamo di 5 la dimensione di'food'
if event.type == MOUSEBUTTONDOWN:
    if event.button == 1:
        print('Hai premuto il tasto sinistro del mouse')
    elif event.button == 3:
        print('Hai premuto il tasto destro del mouse')
    else:
        print("Hai premuto il tasto", event.button)
\end{lstlisting}
\end{frame}

\section{Immagini}

\begin{frame}[fragile]
\frametitle{Immagini}
    Andiamo ora a sostituire il nostro quadrato nero (per il giocatore) e verde (per il cibo) rispettivamente con l'icona di pacman e un pezzo di pizza.
    Imposteremo anche un'immagine di sfondo per il background del nostro gioco.
    
    \begin{block}{pygame.image.load(...)}
        \begin{itemize}
            \item prende come parametro una stringa con il nome dell'immagine.
            \item ricordiamoci che l'immagine che vogliamo usare si deve trovare nella stessa cartella del nostro file python.
        \end{itemize}
    \end{block}
\end{frame}

\begin{frame}[fragile]
\frametitle{Immagini}
    \begin{block}{pygame.image.load(...)}
        \begin{itemize}
            \item Carica un'immagine a partire da un file specificato come parametro.
            \item Prende come parametro una stringa con il nome dell'immagine.
            \item Ricordiamoci che l'immagine che vogliamo usare si deve trovare nella stessa cartella del nostro file python.
        \end{itemize}
    \end{block}
    
    \begin{lstlisting}
    backgroundImage = pygame.image.load('background.png')
    playerImage = pygame.image.load('pacman.png')
    foodImage = pygame.image.load('pizza.png')
    \end{lstlisting}
    \href{https://github.com/ragazzedigitalicesena/slide-2019/raw/master/tex/chapter_19-20/images/background.png}{Download background.png}

    \href{https://github.com/ragazzedigitalicesena/slide-2019/raw/master/tex/chapter_19-20/images/pacman.png}{Download pacman.png}
        
    \href{https://github.com/ragazzedigitalicesena/slide-2019/raw/master/tex/chapter_19-20/images/pizza.png}{Download pizza.png}    
\end{frame}

% pygame.transform.scale
\begin{frame}[fragile]
\frametitle{Immagini}
    \begin{block}{pygame.transform.scale(image, (width, height))}
        \begin{itemize}
            \item Ridimensiona un'immagine
            \item Prende come parametri
                \begin{itemize}
                    \item l'immagine da ridimensionare
                    \item Una tupla (width, height) che rappresenta la larghezza e l'altezza desiderata per quell'immagine.
                \end{itemize}{}
        \end{itemize}
    \end{block}
    
    \begin{lstlisting}
backgroundStretchedImage = pygame.transform.scale(backgroundImage, (WINDOW_WIDTH, WINDOW_HEIGHT))
playerStretchedImage = pygame.transform.scale(playerImage, (PLAYER_WIDTH, PLAYER_HEIGHT))
foodStretchedImage = pygame.transform.scale(foodImage, (foodSize, foodSize))
    \end{lstlisting}
\end{frame}

\begin{frame}[fragile]
\frametitle{Immagini}
    \begin{block}{Cosa faceva windowSurface.blit(...)???}
Disegna un'immagine dentro una superficie (o una superficie dentro una superficie).

Non ci servono piu' ne windowSurface.fill(WHITE) ne i due pygame.draw.rect(...)
    \end{block}
    
    \begin{lstlisting}
# windowSurface.fill(WHITE)
windowSurface.blit(backgroundStretchedImage, windowSurfaceRectangle)
windowSurface.blit(playerStretchedImage, player)
windowSurface.blit(foodStretchedImage, food)
#pygame.draw.rect(windowSurface, BLACK, player)
#pygame.draw.rect(windowSurface, GREEN, food)
\end{lstlisting}
\end{frame}

\begin{frame}{Esempi}
    \begin{center}
        \href{https://raw.githubusercontent.com/ragazzedigitalicesena/slide-2019/master/tex/chapter_19-20/collisionDetection_exercise.py}{Con un solo food e le immagini - Download}
        
        \href{https://raw.githubusercontent.com/ragazzedigitalicesena/slide-2019/master/tex/chapter_19-20/collisionDetection_example.py}{Simile ma con una lista di cibi (foods) ma senza immagini - Download}
    \end{center}
\end{frame}{}

\begin{frame}

\begin{center}
    \bigskip
    Materiale rilasciato con licenza
    
    \textbf{\href{http://creativecommons.org/licenses/by-sa/4.0/}{Creative Commons - Attributions, Share-alike 4.0}}
    
    \medskip
    \includegraphics[height=0.8cm]{images/cc.png}
\end{center}

\end{frame}

\end{document}